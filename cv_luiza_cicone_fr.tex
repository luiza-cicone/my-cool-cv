%% start of file `template_en.tex'.
%% Copyright 2007 Xavier Danaux (xdanaux@gmail.com).
%
% This work may be distributed and/or modified under the
% conditions of the LaTeX Project Public License version 1.3c,
% available at http://www.latex-project.org/lppl/.


\documentclass[10pt,a4paper]{moderncv}

% moderncv themes
\moderncvtheme[green]{classic}                % idem

% character encoding
\usepackage[utf8]{inputenc}                   % replace by the encoding you are using
\usepackage{multicol}


\setlength{\textwidth}{19cm}	
\setlength{\textheight}{28cm}
\setlength{\oddsidemargin}{-1.5cm}
\setlength{\evensidemargin}{-1cm}
\setlength{\topmargin}{-3cm}

% personal data
\firstname{Luiza Maria}
\familyname{CICONE}
\title{Ingénieur systèmes d'information}
\address{}{9 Boulevard Clemenceau}{38100 Grenoble}   % optional, remove the line if not wanted
\mobile{07 81 36 81 44}                    % optional, remove the line if not wanted
%\phone{(312) 413-8265}                      % optional, remove the line if not wanted
%\fax{312 996 1491}                          % optional, remove the line if not wanted
\email{luiza.cicone@gmail.com}                   % optional, remove the line if not wanted
% \emailbis{luiza-maria.cicone@ensimag.fr} % optional, remove the line if not wanted
% \photo[64pt]{portret}                         % '64pt' is the height
% \quote{} 
%\nopagenumbers{}                             % uncomment to suppress automatic page numbering for CVs longer than one page


%----------------------------------------------------------------------------------
%            content
%----------------------------------------------------------------------------------
\begin{document}
\vspace{-1cm}

\maketitle

\vspace{-1cm}

\section{Expériences}
\subsection{Stages}
\cventry{2014 (6 mois)}
{Développement Web et iOS}
{Orange Business Services IT\&L@bs}{Montbonnot Saint-Martin}{}
{Conception et implémentation d'une solution de gestion de questionnaires (méthodologie \textbf{SCRUM}) :}
\cvlistitem{\small Serveur \textbf{node.js} avec base de données \textbf{MongoDB} délivrant les questionnaires via une API REST JSON}
\cvlistitem{\small Application web cliente d'administration de questionnaires avec \textbf{Angular.js}}
\cvlistitem{\small Application iPad cliente de réponse aux questionnaires}
\cventry{2013 (6 mois)}
{Développement Web}
{Laboratoire Informatique de Grenoble, Ingénierie de l'Interaction Homme-Machine}{}{}{Conception et implémentation d'une application web de support au processus de développement centré utilisateur réalisée avec \textbf{node.js}.}
%Outil qui guide les concepteurs débutants dans l'analyse et la conception de systèmes intégrant des interfaces homme-machine. 
\cventry{2012 (3 mois)}{Développement Android}{Bassetti}{Grenoble}{}{Conception et implémentation d'une application tablette \textbf{Android} assistant les ingénieurs méthodes à recueillir et analyser des données terrain pour l'optimisation des chaînes de production.}   
\cventry{2011 (2 mois)}{Développement iOS}{Institut de Recherche en Informatique (IRIT)}{Toulouse}{}{Refonte et optimisation d'une application iPad existante (ADRIA-Nutri-Educ) par la conception du modèle de données et son intégration avec \textbf{Core Data}.}   

%\subsection{Projets scolaires}
%\cventry{2012 (1 mois)}{Projet Logiciel - Application web}{}{}{}{Conception et implémentation d'une plate-forme web en \textbf{Java Servlets et JSP)} pour un service ambulancier.}
%\cventry{2012 (1 mois)}{Projet Génie Logiciel - Compilateur}{}{}{}{Implémentation d'un compilateur en Ada pour un langage orienté-objet proche de Java}
%\cventry{2011 (1 mois)}{Projet Bases de Données}{}{}{}{Conception d'une base de données Oracle et implémentation d'une interface utilisateur en Java pour un restaurant}
%\cventry{2011}{Pratique du système d'exploitation}{}{}{}{Réalisation d'extensions sur un noyau Unix (interruptions, entrées/sorties..)}

\subsection{Projets à initiative personelle}
\cventry{2013}{JeuxVideoMobile}{}{}{}{Réalisation d'une application iOS de news pour \href{http://jeuxvideomobile.com}{jeuxvideomobile.com}}
%\href{https://itunes.apple.com/sr/app/jeux-video-mobile-test-solution/id622853455?mt=8}
%{https://itunes.apple.com/sr/app/jeux-video-mobile-test-solution/id622853455?mt=8}}
\cventry{2012}{Grand Cercle Mobile}{}{}{}{Réalisation d'une application iOS pour le Cercle des Élèves de Grenoble INP (lien iTunes : \href{http://goo.gl/zv6I4j}{http://goo.gl/zv6I4j)}}
\cventry{2011}{Développement iOS}{Cronian Labs}{Bucarest}{}{Implémentation d'un nouveau module de traitement d'image pour l'application iPhone SkinScan.}

\vspace{-1.2cm}

\setlength{\columnsep}{2cm}
\begin{multicols}{2}
\section{Compétences} 
\cvline{Programmation}{Java, C, Objective-C, Ada, C++, Delphi}
\cvline{Web}{HTML5, CSS3, JavaScript, Angular.js, D3.js}
\cvline{Base de données}{MySQL, Oracle, SQLite, MongoDB}
\cvline {Outils, méthodes}{git, subversion, LaTeX, UML, Agile (SCRUM)}

\vfill

\columnbreak
\section{Langues}
\cvlanguage{Roumain}{Langue maternelle}{}
\cvlanguage{Français}{Courant, \textnormal{pratique quotidienne}}{}
\cvlanguage{Anglais}{Courant, \textnormal{TOEIC 990/990}}{}
\cvlanguage{Espagnol}{Notions}{}
\end{multicols}

\vspace{-.8cm}

\section{Diplômes et études}
\cventry{2011-présent}{Formation d'ingénieur en double-diplôme}{}{Ensimag-Grenoble INP}{France}{École nationale supérieure d'informatique et de mathématiques appliquées\\Filière Ingénierie des Systèmes d'Information (ISI)}
\cventry{2009-2013}{Licence en informatique}{}{Université POLITEHNICA de Bucarest (UPB)}{Roumanie}{Faculté d'Ingénierie en Langues Étrangères (FILS), spécialisation informatique, filière francophone}
\cventry{2005-2009}{Baccalauréat scientifique}{Lycée d'Informatique 'Tudor Vianu'}{Bucarest}{Roumanie}{Filière Mathématique-Informatique} 

\vspace{-.2cm}

\section{Clubs et associations}
\cvline{2012 et 2013}{\textbf{Cercle des élèves de Grenoble INP}, membre du pôle communication et responsable informatique.}
\cvlistitem{\small réalisation du site web : \href{http://grandcercle.org}{http://grandcercle.org}}
\cvline{2009--2011}{\textbf{Association des étudiants et des anciens de la filière francophone (ASAFF)}}
\cvline{2010--2011}{\textbf{RobotiqueFF}, club robotique de l'Université POLITEHNICA de Bucarest}
\cvlistitem{\small 3ème prix du Concours National RoboChallenge, Bucarest 2011}
\cvlistitem{\small Participation au Concours International RobotChallenge, Vienne 2011}
\cvline{2009--2011}{\textbf{Impulsion}, club d'improvisation théâtrale francophone}
% \cvlistitem{\small stage formation à Timisoara 2009 (Richard Pineault et Laurent Vincent)}
\cvlistitem{\small 1er prix du Festival International Francophone d'Improvisation Théâtrale Universitaire Timisoara (FIFITUT) 2010}

\vspace{-.3cm}
\section{Loisirs}
\cvdoubleitem{Sport}{\small basket-ball, ski, running trail, vélo}{Musique}{\small piano et chant}

\end{document}